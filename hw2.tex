\documentclass{article}
\usepackage{amsmath}
\usepackage{enumerate}
\usepackage{fancyhdr} % Required for custom headers
\usepackage{lastpage} % Required to determine the last page for the footer
\usepackage{extramarks} % Required for headers and footers
\usepackage[usenames,dvipsnames]{color} % Required for custom colors
\usepackage{graphicx} % Required to insert images
\usepackage[tight,footnotesize]{subfigure} % Required for subfig
\usepackage{caption} % Required for subfig
\usepackage{hyperref} % Required for url
\usepackage{listings} % Required for insertion of code
\usepackage{courier} % Required for the courier font
\usepackage{lipsum} % Used for inserting dummy 'Lorem ipsum' text into the template
\topmargin=-0.45in
\evensidemargin=0in
\oddsidemargin=0in
\textwidth=6.5in
\textheight=9.0in
\headsep=0.25in
\linespread{1.1} % Line spacing
\pagestyle{fancy}
\lhead{\hmwkAuthorName} % Top left header
% \chead{\hmwkClass\ (\hmwkClassInstructor\ \hmwkClassTime): \hmwkTitle} % Top center head
\chead{\hmwkClass\ : \hmwkTitle} % Top center head
\rhead{\firstxmark} % Top right header
\lfoot{\lastxmark} % Bottom left footer
\cfoot{} % Bottom center footer
\rfoot{Page\ \thepage\ of\ \protect\pageref{LastPage}} % Bottom right footer
\renewcommand\headrulewidth{0.4pt} % Size of the header rule
\renewcommand\footrulewidth{0.4pt} % Size of the footer rule
\setlength\parindent{0pt} % Removes all indentation from paragraphs

% Define floor and ceiling
\def\lc{\left\lceil}   
\def\rc{\right\rceil}
\def\lf{\left\lfloor}   
\def\rf{\right\rfloor}

% Set your language 
%\lstset{language=Java}
\definecolor{codegreen}{rgb}{0,0.6,0}
\definecolor{codegray}{rgb}{0.5,0.5,0.5}
\definecolor{codepurple}{rgb}{0.58,0,0.82}
\definecolor{backcolour}{rgb}{0.95,0.95,0.92}
 
\lstdefinestyle{mystyle}{
    backgroundcolor=\color{backcolour},   
    commentstyle=\color{codegreen},
    keywordstyle=\color{magenta},
    numberstyle=\tiny\color{codegray},
    stringstyle=\color{codepurple},
    basicstyle=\footnotesize,
    breakatwhitespace=false,         
    breaklines=true,                 
    captionpos=b,                    
    keepspaces=true,                 
    numbers=left,                    
    numbersep=8pt,                  
    showspaces=false,                
    showstringspaces=false,
    showtabs=false,                  
    tabsize=2
}
\lstset{style=mystyle}

% Header and footer for when a page split occurs within a problem environment
\newcommand{\enterProblemHeader}[1]{
\nobreak\extramarks{#1}{#1 continued on next page\ldots}\nobreak
\nobreak\extramarks{#1 (continued)}{#1 continued on next page\ldots}\nobreak
}

% Header and footer for when a page split occurs between problem environments
\newcommand{\exitProblemHeader}[1]{
\nobreak\extramarks{#1 (continued)}{#1 continued on next page\ldots}\nobreak
\nobreak\extramarks{#1}{}\nobreak
}

\setcounter{secnumdepth}{0} % Removes default section numbers
\newcounter{homeworkProblemCounter} % Creates a counter to keep track of the number of problems

\newcommand{\homeworkProblemName}{}
\newenvironment{homeworkProblem}[1][Problem \arabic{homeworkProblemCounter}]{ % Makes a new environment called homeworkProblem which takes 1 argument (custom name) but the default is "Problem #"
\stepcounter{homeworkProblemCounter} % Increase counter for number of problems
\renewcommand{\homeworkProblemName}{#1} % Assign \homeworkProblemName the name of the problem
\section{\homeworkProblemName} % Make a section in the document with the custom problem count
\enterProblemHeader{\homeworkProblemName} % Header and footer within the environment
}{
\exitProblemHeader{\homeworkProblemName} % Header and footer after the environment
}

\newcommand{\problemAnswer}[1]{ % Defines the problem answer command with the content as the only argument
\noindent\framebox[\columnwidth][c]{\begin{minipage}{0.98\columnwidth}#1\end{minipage}} % Makes the box around the problem answer and puts the content inside
}

\newcommand{\homeworkSectionName}{}
\newenvironment{homeworkSection}[1]{ % New environment for sections within homework problems, takes 1 argument - the name of the section
\renewcommand{\homeworkSectionName}{#1} % Assign \homeworkSectionName to the name of the section from the environment argument
\subsection{\homeworkSectionName} % Make a subsection with the custom name of the subsection
\enterProblemHeader{\homeworkProblemName\ [\homeworkSectionName]} % Header and footer within the environment
}{
\enterProblemHeader{\homeworkProblemName} % Header and footer after the environment
}

\newlength{\tabcont}

\newcommand{\tab}[1]{%
\settowidth{\tabcont}{#1}%
\ifthenelse{\lengthtest{\tabcont < .25\linewidth}}%
{\makebox[.25\linewidth][l]{#1}\ignorespaces}%
{\makebox[.5\linewidth][l]{\color{red} #1}\ignorespaces}%
}%
%----------------------------------------------------------------------------------------
%	NAME AND CLASS SECTION
%----------------------------------------------------------------------------------------

\newcommand{\hmwkTitle}{Homework\ \#2} % Assignment title
\newcommand{\hmwkDueDate}{Monday,\ January\ 1,\ 2012} % Due date
\newcommand{\hmwkClass}{Fundamental Algorithms} % Course/class
\newcommand{\hmwkClassTime}{} % Class/lecture time
\newcommand{\hmwkClassInstructor}{Prof. Joel Spencer} % Teacher/lecturer
\newcommand{\hmwkAuthorName}{Songxiao Zhang, N10224459} % Your name

%----------------------------------------------------------------------------------------
%	TITLE PAGE
%----------------------------------------------------------------------------------------

\title{
\textmd{\textbf{\hmwkClass:\ \hmwkTitle}}\\
}
\author{\textbf{\hmwkAuthorName}}

\begin{document}

\maketitle

%----------------------------------------------------------------------------------------
%	PROBLEM 1
%----------------------------------------------------------------------------------------
\begin{homeworkProblem}
The pattern of the book
\begin{lstlisting}[frame=single]
 [13, 19, 9, 5, 12, 8, 7, 4, 11, 2, 6, 10]    i = 0, j = 1
 [13, 19, 9, 5, 12, 8, 7, 4, 11, 2, 6, 10]    i = 0, j = 2
 [9, 19, 13, 5, 12, 8, 7, 4, 11, 2, 6, 10]    i = 1, j = 3
 [9, 5, 13, 19, 12, 8, 7, 4, 11, 2, 6, 10]    i = 2, j = 4
 [9, 5, 13, 19, 12, 8, 7, 4, 11, 2, 6, 10]    i = 2, j = 5
 [9, 5, 8, 19, 12, 13, 7, 4, 11, 2, 6, 10]    i = 3, j = 6
 [9, 5, 8, 7, 12, 13, 19, 4, 11, 2, 6, 10]    i = 4, j = 7
 [9, 5, 8, 7, 4, 13, 19, 12, 11, 2, 6, 10]    i = 5, j = 8
 [9, 5, 8, 7, 4, 13, 19, 12, 11, 2, 6, 10]    i = 5, j = 9
 [9, 5, 8, 7, 4, 2, 19, 12, 11, 13, 6, 10]    i = 6, j = 10
 [9, 5, 8, 7, 4, 2, 6, 12, 11, 13, 19, 10]    i = 7, j = 11
 [9, 5, 8, 7, 4, 2, 6, 10, 11, 13, 19, 12]    i = 7
 return 7
\end{lstlisting}
\end{homeworkProblem}

%----------------------------------------------------------------------------------------
%	PROBLEM 2
%----------------------------------------------------------------------------------------
\begin{homeworkProblem}
\begin{enumerate}[1.]
    \item $1023 - 1 = 1022$ comparisons between $A(1023)$ and all the other $A(j)$
    \item $partition(A, 1, 1023)$ would end up at 512, QUICKSORT(A, 1, 511) and QUICKSORT(A, 513, 1023). 
          The number of comparisons would be $2 * (511 - 1) = 1022$
    \item $partition(A, 1, 511)$ would end up at 256 $\ldots$
\end{enumerate}
The total number of comparisons would be 
$$(N-1) + (N - 1 - 2) + (N - 1 - 2 - 2^2) + \ldots + (N+1)/2 $$
$$= 1022 + (1022 - 2) + (1022 - 2 - 2^2) + (1022 - 2 - 2^3) + \ldots + 512 $$
$$= 1022 \times log_2^{(1023 + 1)} - 2 \times (log_2^{(1023 + 1)} - 1) - \ldots - 512 $$
$$= 1022 \times 10 - 2 \times 9 - 4 \times 8 - \ldots - 1 \times 512$$
$$= 1022 \times 10 - 0 - 2- 6- 14 - 30 - 62 - 126 - 254 - 510$$
$$= 8194 $$
\end{homeworkProblem}

%----------------------------------------------------------------------------------------
%	PROBLEM 3
%----------------------------------------------------------------------------------------
\begin{homeworkProblem}
$b$ has to compare with $c$ and $d$ with 3 possible positions;
$e$ needs to compare with all 4 with 5 possible positions. Here, we can expand the decision tree
by depth-first search, giving each case a path in a linked list. \\
$b$ needs to compare 2 times: $b<c$, $b>d$, $b>c -> b<d$ or $b>c -> b>d$. \\
$e$ needs to compare 3 times: $e<a$, $e>d$, $e<b -> e<a$, $e<b -> e>a$, $e>c -> e>d$, $e>c -> e<d$, 
$e>b -> e>c -> e>d$ or $e<c -> e<a -> e<b$. \\
There is no path to finish sorting in 4 further steps (another 4 levels in the decision tree). 
At least 5 further steps needed. 

% Shortest path is 2. 
% 2 further paths. $b>d --> e>b => acdbe$
% 3 further paths. $b>c --> b>d --> e>b => acdbe$

\end{homeworkProblem}

%----------------------------------------------------------------------------------------
%	PROBLEM 4
%----------------------------------------------------------------------------------------
\begin{homeworkProblem}

The worst-case is $a$ takes 4 comparisons for $a<x$, where $x \in \left( b, c, d, e\right) $ when $a$ is the largest. 
Similar for $b$ takes 3. In total for worst case, $4+3+2+1 = 10$. \\

The other way to think about this is decision tree. 5 questions can only generate $2^5 = 32$ decisions, which is not enough for 40 decisions as there is 40 leaves in the subtree. 

\end{homeworkProblem}

%----------------------------------------------------------------------------------------
%	PROBLEM 5
%----------------------------------------------------------------------------------------
\begin{homeworkProblem}
Counting array \\
0,0,0,0,0,0,1  \\
1,0,0,0,0,0,1  \\
1,0,1,0,0,0,1   \\
1,0,2,0,0,0,1  \\
2,0,2,0,0,0,1  \\
2,1,2,0,0,0,1  \\
2,1,2,1,0,0,1  \\
2,1,2,1,1,0,1  \\
2,1,2,1,1,0,2   \\
2,2,2,1,1,0,2   \\
2,2,2,2,1,0,2   \\
2,4,6,8,9,9,11  \\

New array    \\
0,0,0,0,0,0,0,0,0,0,0   \\
0,0,1,0,0,0,0,0,0,0,0   \\
0,0,1,1,0,0,0,0,0,0,0   \\
0,0,1,1,2,0,0,0,0,0,0   \\
0,0,1,1,2,2,0,0,0,0,0   \\
0,0,1,1,2,2,3,0,0,0,0   \\
0,0,1,1,2,2,3,3,0,0,0   \\
0,0,1,1,2,2,3,3,4,0,0   \\
0,0,1,1,2,2,3,3,4,6,0   \\
0,0,1,1,2,2,3,3,4,6,6   \\

\end{homeworkProblem}

%----------------------------------------------------------------------------------------
%	PROBLEM 6
%----------------------------------------------------------------------------------------
\begin{homeworkProblem}
\texttt{HEAP-EXTRACT-MAX} is built upon \texttt{MAX-HEAPIFY} plus 3 operations on delete 
the last node, exhanging first and last, and truncate the size. Running time is $O(lgN)$. 

Actually, the third largest entry can only exist in the 2nd or 3rd level. Simply check 
\begin{lstlisting}[frame=single]
if A[2]>A[3] 
    if A[3] > A[2].left && A[3] > A[2].right
        return 3;
    else if A[3] < A[2].left
            return 4
         else return 5
if A[3]>A[2] 
    if A[2] > A[3].left && A[2] > A[3].right
        return 2;
    else if A[2] < A[3].left
            return 6
         else return 7
\end{lstlisting}
The running time is at most $2+3 = 5$. Much smaller than $O(lgN)$ when $N$ gets large. 

\end{homeworkProblem}





% \begin{enumerate}[a.]
%     \item 
% \end{enumerate}
\end{document}